\documentclass[10pt, a4paper, twocolumn]{article}
\usepackage{multicol}

\setlength{\columnsep}{0.5cm}

\begin{document}
\footnotesize
\section*{Pure Python}
\subsection*{Types}
\begin{tabular}{ p{0.45\columnwidth} p{0.45\columnwidth} }
  a = 2 & integer\\
  b = 5.0 & float\\
  c = 8.3e5 & exponential\\
  d = 1.5 + 0.5j & complex\\
  e = 3 $>$ 4 & boolean\\
  f = 'word' & string\\
\end{tabular}

\subsection*{Lists}
\begin{tabular}{ p{0.45\columnwidth} p{0.45\columnwidth} }
  a = ['red', 'blue', 'green'] & manually initialization\\
  b = range(5) & initialization through a function\\
  c = [nu**2 for nu in b] & initialize through list comprehension\\
  d = [nu**2 for nu in b if b $<$ 3] & list comprehension with condition\\
  e = c[0] & access element\\
  f = e[1: 2] & access a slice of the list\\
  g = ['re', 'bl'] + ['gr'] & list concatenation\\
  h = ['re'] * 5 & repeat a list\\[1pt]
  ['re', 'bl'].index('re') & returns index of 're'\\
  're' in ['re', 'bl'] & true if 're' in list\\
  sorted([3, 2, 1]) & returns sorted list\\
  z = ['red'] + ['green', 'blue'] & list concatenation\\
\end{tabular}

\subsection*{Dictionaries}
\begin{tabular}{ p{0.45\columnwidth} p{0.45\columnwidth} }
  a = \{'red': 'rouge', 'blue': 'bleu', 'green': 'vert'\} & dictionary\\
  b = a['red'] & translate item\\
  c = [value for key, value in b.items()] & loop through contents\\
  d = a.get('yellow', 'no translation found') & return default\\
\end{tabular}

\subsection*{Strings}
\begin{tabular}{ p{0.45\columnwidth} p{0.45\columnwidth} }
  a = 'red' & assignment\\
  char = a[2] & access individual characters\\
  'red ' + 'blue' & string concatenation\\
  '1, 2, three'.split(',') & split string into list\\
  '.'.join(['1', '2', 'three']) & concatenate list into string\\
\end{tabular}

\subsection*{Operators}
\begin{tabular}{ p{0.45\columnwidth} p{0.45\columnwidth} }
  a = 2 & assignment\\
  a += 1 (*=, /=) & change and assign\\
  3 + 2 & addition\\
  3 / 2 & integer division (python2) or float division (python3)\\
  3 // 2 & integer division\\
  3 * 2 & multiplication\\
  3 ** 2 & exponent\\
  3 \% 2 & remainder\\
  abs() & absolute value\\
  1 == 1 & equal\\
  2 $>$ 1 & larger\\
  2 $<$ 1 & smaller\\
  1 != 2 & not equal\\
  1 != 2 and 2 $<$ 3 & logical AND\\
  1 != 2 or 2 $<$ 3 & logical OR\\
  not 1 == 2 & logical NOT\\
  a in b & test if a is in b\\
  a is b & test if objects point to the same memory (id)\\[1pt]
\end{tabular}

\subsection*{Control Flow}
\begin{tabular}{ p{0.45\columnwidth} p{0.45\columnwidth} }
  \begin{minipage}[t]{\columnwidth}
    \textit{if/elif/else}
\begin{verbatim}
a, b = 1, 2
if a + b == 3:
    print 'True'
elif a + b == 1:
    print 'False'
else:
    print '?'
\end{verbatim}
  \end{minipage}
  &
    \begin{minipage}[t]{\columnwidth}
      \textit{for}
\begin{verbatim}
a = ['red', 'blue',
     'green']
for color in a:
    print color
\end{verbatim}
    \end{minipage}\\
  \rule{0pt}{0.5cm}

  \begin{minipage}[t]{\columnwidth}
    \textit{while}
\begin{verbatim}
number = 1
while number < 10:
    print number
    number += 1
\end{verbatim}
  \end{minipage}
  &
    \begin{minipage}[t]{\columnwidth}
      \textit{break}
\begin{verbatim}
number = 1
while True:
    print number
    number += 1
    if number > 10:
        break
\end{verbatim}
    \end{minipage}\\
  \rule{0pt}{0.5cm}

  \begin{minipage}[t]{\columnwidth}
    \textit{continue}
\begin{verbatim}
for i in range(20):
    if i % 2 == 0:
        continue
    print i
\end{verbatim}
  \end{minipage}
  &
\end{tabular}

\subsection*{Functions, Classes, Generators, Decorators}
\begin{tabular}{ p{0.45\columnwidth} p{0.45\columnwidth} }
  \begin{minipage}[t]{\columnwidth}
    \textit{Function}
\begin{verbatim}
def myfunc(a1, a2):
    return x

x = my_function(a1,a2)
\end{verbatim}
  \end{minipage}
  &
    \begin{minipage}[t]{\columnwidth}
      \textit{Class}
\begin{verbatim}
class Point(object):
    def __init__(self, x):
        self.x = x
    def __call__(self):
        print self.x

x = Point(3)
\end{verbatim}
    \end{minipage}\\
  \rule{0pt}{0.5cm}

  \begin{minipage}[t]{\columnwidth}
    \textit{Generator}
\begin{verbatim}
def firstn(n):
    num = 0
    while num < n:
        yield num
        num += 1

x = [for i in firstn(10)]
\end{verbatim}
  \end{minipage}
  &
    \begin{minipage}[t]{\columnwidth}
      \textit{Decorator}
\begin{verbatim}
class myDecorator(object):
    def __init__(self, f):
        self.f = f
    def __call__(self):
        print "call"
        self.f()

@myDecorator
def my_funct():
    print 'func'

my_func()
\end{verbatim}
    \end{minipage}
\end{tabular}


\section*{NumPy}
\subsection*{array initialization}
\begin{tabular}{ p{0.45\columnwidth} p{0.5\columnwidth} }
  np.array([2, 3, 4]) & direct initialization\\
  np.empty(20, dtype=np.float32) & single precision array with 20 entries\\
  np.zeros(200) & initialize 200 zeros\\
  np.ones((3,3), dtype=np.int32) & 3 x 3 integer matrix with ones\\
  np.eye(200) & ones on the diagonal\\
  np.zeros\_like(a) & returns array with zeros and the same shape as a\\
  np.linspace(0., 10., 100) & 100 points from 0 to 10\\
  np.arange(0, 100, 2) & points from 0 to $<$100 with step width 2\\
  np.logspace(-5, 2, 100) & 100 logarithmically spaced points between 1e-5 and 1e2\\
  np.copy(a) & copy array to new memory\\
\end{tabular}

\subsection*{reading/ writing files}
\begin{tabular}{ p{0.45\columnwidth} p{0.5\columnwidth} }
  np.fromfile(fname/object, dtype=np.float32, count=5) & read binary data from file\\
  np.loadtxt(fname/object, skiprows=2, delimiter=',') & read ascii data from file
\end{tabular}

\subsection*{array properties and operations}
\begin{tabular}{ p{0.45\columnwidth} p{0.5\columnwidth} }
  a.shape & a tuple with the lengths of each axis\\
  len(a) & length of axis 0\\
  a.ndim & number of dimensions (axes)\\
  a.sort(axis=1) & sort array along axis\\
  a.flatten() & collapse array to one dimension\\
  a.conj() & return complex conjugate\\
  a.astype(np.int16) & cast to integer\\
  np.argmax(a, axis=2) & return index of maximum along a given axis\\
  np.cumsum(a) & return cumulative sum\\
  np.any(a) & True if any element is True\\
  np.all(a) & True if all elements are True\\
  np.argsort(a, axis=1) & return sorted index array along axis\\
\end{tabular}

\subsection*{indexing}
\begin{tabular}{ p{0.45\columnwidth} p{0.5\columnwidth} }
  a = np.arange(100) & initialization with 0 - 99\\
  a[: 3] = 0 & set the first three indices to zero\\
  a[1: 5] = 1 & set indices 1-4 to 1\\
  a[start:stop:step] & general form of indexing/slicing\\
  a[None, :] & transform to column vector\\
  a[[1, 1, 3, 8]] & return array with values of the indices\\
  a = a.reshape(10, 10) & transform to 10 x 10 matrix\\
  a.T & return transposed view\\
  np.transpose(a, (2, 1, 0)) & transpose array to new axis order\\
  a[a $<$ 2] & returns array that fulfills elementwise condition\\
\end{tabular}

\subsection*{boolean arrays}
\begin{tabular}{ p{0.6\columnwidth} p{0.35\columnwidth} }
  a $<$ 2 & returns array with boolean values\\
  np.logical\_and(a $<$ 2, b $>$ 10) & elementwise logical and\\
  np.logical\_or(a $<$ 2, b $>$ 10) & elementwise logical or\\
  -a & invert boolean array\\
  np.invert(a) & invert boolean array\\
\end{tabular}

\subsection*{elementwise operations and math functions}
\begin{tabular}{ p{0.3\columnwidth} p{0.65\columnwidth} }
  a * 5 & multiplication with scalar\\
  a + 5 & addition with scalar\\
  a + b & addition with array b\\
  a / b & division with b (np.NaN for division by zero)\\
  np.exp(a) & exponential (complex and real)\\
  np.sin(a) & sine\\
  np.cos(a) & cosine\\
  np.arctan2(y,x) & arctan(y/x)\\
  np.arcsin(x) & arcsin\\
  np.radians(a) & degrees to radians\\
  np.degrees(a) & radians to degrees\\
  np.var(a) & variance of array\\
  np.std(a, axis=1) & standard deviation\\
\end{tabular}

\subsection*{inner / outer products}
\begin{tabular}{ p{0.5\columnwidth} p{0.45\columnwidth} }
  np.dot(a, b) & inner matrix product: a\_mi b\_in \\
  np.einsum('ijkl,klmn-$>$ijmn', a, b) & einstein summation convention\\
  np.sum(a, axis=1) & sum over axis 1\\
  np.abs(a) & return array with absolute values\\
  a[None, :] + b[:, None] & outer sum\\
  a[None, :] * b[None, :] & outer product\\
  np.outer(a, b) & outer product\\
  np.sum(a * a.T) & matrix norm\\
\end{tabular}

\subsection*{interpolation, integration}
\begin{tabular}{ p{0.5\columnwidth} p{0.45\columnwidth} }
  np.trapz(y, x=x, axis=1) & integrate along axis 1\\
  np.interp(x, xp, yp) & interpolate function xp, yp at points x\\
\end{tabular}

\subsection*{fft}
\begin{tabular}{ p{0.5\columnwidth} p{0.45\columnwidth} }
  np.fft.fft(y) & complex fourier transform of y\\
  np.fft.fftfreqs(len(y)) & fft frequencies for a given length\\
  np.fft.fftshift(freqs)   & shifts zero frequency to the middle\\
  np.fft.rfft(y) & real fourier transform of y\\
  np.fft.rfftfreqs(len(y)) & real fft frequencies for a given length\\
\end{tabular}

\subsection*{rounding}
\begin{tabular}{ p{0.5\columnwidth} p{0.45\columnwidth} }
  np.ceil(a) & rounds to nearest upper int\\
  np.floor(a) & rounds to nearest lower int\\
  np.round(a) & rounds to neares int\\
\end{tabular}

\subsection*{random variables}
\begin{tabular}{ p{0.6\columnwidth} p{0.35\columnwidth} }
  np.random.normal(loc=0, scale=2, size=100) & 100 normal distributed random numbers\\
  np.random.seed(23032) & resets the seed value\\
  np.random.rand(200) & 200 random numbers in [0, 1)\\
  np.random.uniform(1, 30, 200)         & 200 random numbers in [1, 30)\\
  np.random.random\_integers(1, 15, 300) & 300 random integers between [1, 10]\\
\end{tabular}

\newpage
\section*{Matplotlib}
\subsection*{figures and axes}
\begin{tabular}{ p{0.6\columnwidth} p{0.35\columnwidth} }
  fig = plt.figure(figsize=(5, 2), facecolor='black') & initialize figure\\
  ax = fig.add\_subplot(3, 2, 2) & add second subplot in a 3 x 2 grid\\
  fig, axes = plt.subplots(5, 2, figsize=(5, 5)) & return fig and array of axes in a 5 x 2 grid\\
  ax = fig.add\_axes([left, bottom, width, height]) & manually add axes at a certain position\\
\end{tabular}

\subsection*{figures and ax properties}
\begin{tabular}{ p{0.6\columnwidth} p{0.35\columnwidth} }
  fig.suptitle('title') & big figure title\\
  fig.subplots\_adjust(bottom=0.1, right=0.8, top=0.9, wspace=0.2, hspace=0.5) & adjust subplot positions\\
  fig.tight\_layout(pad=0.1,h\_pad=0.5, w\_pad=0.5, rect=None) & adjust subplots to fit perfectly into fig\\
  ax.set\_xlabel() & set xlabel\\
  ax.set\_ylabel() & set ylabel\\
  ax.set\_xlim(1, 2) & sets x limits\\
  ax.set\_ylim(3, 4) & sets y limits\\
  ax.set\_title('blabla') & sets the axis title\\
  ax.set(xlabel='bla') & set multiple parameters at once\\ 
  ax.legend(loc='upper center') & activate legend\\
  ax.grid(True, which='both') & activate grid\\
  bbox = ax.get\_position() & returns the axes bounding box\\
  bbox.x0 + bbox.width & bounding box parameters\\
\end{tabular}

\subsection*{plotting routines}
\begin{tabular}{ p{0.6\columnwidth} p{0.35\columnwidth} }
  ax.plot(x,y, '-o', c='red', lw=2, label='bla') & plots a line\\
  ax.scatter(x,y, s=20, c=color) & scatter plot\\
  ax.pcolormesh(xx,yy,zz, shading='gouraud') & fast colormesh function\\
  ax.colormesh(xx,yy,zz, norm=norm) & slower colormesh function\\
  ax.contour(xx,yy,zz, cmap='jet') & contour line plot\\
  ax.contourf(xx,yy,zz, vmin=2, vmax=4) & filled contours plot\\
  n, bins, patch = ax.hist(x, 50) & histogram\\
  ax.imshow(matrix, origin='lower', extent=(x1, x2, y1, y2)) & show image\\
  ax.specgram(y, FS=0.1, noverlap=128, scale='linear') & plot a spectrogram\\
\end{tabular}


\end{document}
