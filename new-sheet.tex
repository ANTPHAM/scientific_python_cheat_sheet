\documentclass[]{article}
\usepackage[T1]{fontenc}
\usepackage{lmodern}
\usepackage{amssymb,amsmath}
\usepackage{ifxetex}
\usepackage{fixltx2e} % provides \textsubscript
% use microtype if available
\IfFileExists{microtype.sty}{\usepackage{microtype}}{}
\ifnum 0\ifxetex 1\fi=0 % if pdftex
  \usepackage[utf8]{inputenc}
\else % if xelatex
  \usepackage{fontspec}
  \ifxetex
    \usepackage{xltxtra,xunicode}
  \fi
  \defaultfontfeatures{Mapping=tex-text,Scale=MatchLowercase}
  \newcommand{\euro}{€}
\fi
\usepackage{color}
\usepackage{fancyvrb}
\DefineShortVerb[commandchars=\\\{\}]{\|}
\DefineVerbatimEnvironment{Highlighting}{Verbatim}{commandchars=\\\{\}}
% Add ',fontsize=\small' for more characters per line
\newenvironment{Shaded}{}{}
\newcommand{\KeywordTok}[1]{\textcolor[rgb]{0.00,0.44,0.13}{\textbf{{#1}}}}
\newcommand{\DataTypeTok}[1]{\textcolor[rgb]{0.56,0.13,0.00}{{#1}}}
\newcommand{\DecValTok}[1]{\textcolor[rgb]{0.25,0.63,0.44}{{#1}}}
\newcommand{\BaseNTok}[1]{\textcolor[rgb]{0.25,0.63,0.44}{{#1}}}
\newcommand{\FloatTok}[1]{\textcolor[rgb]{0.25,0.63,0.44}{{#1}}}
\newcommand{\CharTok}[1]{\textcolor[rgb]{0.25,0.44,0.63}{{#1}}}
\newcommand{\StringTok}[1]{\textcolor[rgb]{0.25,0.44,0.63}{{#1}}}
\newcommand{\CommentTok}[1]{\textcolor[rgb]{0.38,0.63,0.69}{\textit{{#1}}}}
\newcommand{\OtherTok}[1]{\textcolor[rgb]{0.00,0.44,0.13}{{#1}}}
\newcommand{\AlertTok}[1]{\textcolor[rgb]{1.00,0.00,0.00}{\textbf{{#1}}}}
\newcommand{\FunctionTok}[1]{\textcolor[rgb]{0.02,0.16,0.49}{{#1}}}
\newcommand{\RegionMarkerTok}[1]{{#1}}
\newcommand{\ErrorTok}[1]{\textcolor[rgb]{1.00,0.00,0.00}{\textbf{{#1}}}}
\newcommand{\NormalTok}[1]{{#1}}
\ifxetex
  \usepackage[setpagesize=false, % page size defined by xetex
              unicode=false, % unicode breaks when used with xetex
              xetex]{hyperref}
\else
  \usepackage[unicode=true]{hyperref}
\fi
\hypersetup{breaklinks=true,
            bookmarks=true,
            pdfauthor={},
            pdftitle={},
            colorlinks=true,
            urlcolor=blue,
            linkcolor=magenta,
            pdfborder={0 0 0}}
\setlength{\parindent}{0pt}
\setlength{\parskip}{6pt plus 2pt minus 1pt}
\setlength{\emergencystretch}{3em}  % prevent overfull lines
\setcounter{secnumdepth}{0}

\author{}
\date{}

\begin{document}

\section{Scientific Python Cheatsheet}

\subsection{Pure Python}

\subsubsection{Types}

\begin{Shaded}
\begin{Highlighting}[]
\NormalTok{a = }\DecValTok{2} \CommentTok{# integer}
\NormalTok{b = }\FloatTok{5.0} \CommentTok{# float}
\NormalTok{c = }\FloatTok{8.3e5} \CommentTok{# exponential}
\NormalTok{d = }\FloatTok{1.5} \NormalTok{+}\OtherTok{ 0.5j} \CommentTok{# complex}
\NormalTok{e = }\DecValTok{3} \NormalTok{> }\DecValTok{4} \CommentTok{# boolean}
\NormalTok{f = }\StringTok{"word"} \CommentTok{# string}
\end{Highlighting}
\end{Shaded}

\subsubsection{Lists}

\begin{Shaded}
\begin{Highlighting}[]
\NormalTok{a = [}\StringTok{"red"}\NormalTok{, }\StringTok{"blue"}\NormalTok{, }\StringTok{"green"}\NormalTok{] }\CommentTok{# manually initialization}
\NormalTok{b = }\DataTypeTok{range}\NormalTok{(}\DecValTok{5}\NormalTok{) }\CommentTok{# initialization through a function}
\NormalTok{c = [nu**}\DecValTok{2} \KeywordTok{for} \NormalTok{nu in b] }\CommentTok{# initialize through list comprehension}
\NormalTok{d = [nu**}\DecValTok{2} \KeywordTok{for} \NormalTok{nu in b }\KeywordTok{if} \NormalTok{b < }\DecValTok{3}\NormalTok{] }\CommentTok{# list comprehension withcondition}
\NormalTok{e = c[}\DecValTok{0}\NormalTok{] }\CommentTok{# access element}
\NormalTok{f = e[}\DecValTok{1}\NormalTok{: }\DecValTok{2}\NormalTok{] }\CommentTok{# access a slice of the list}
\NormalTok{g = [}\StringTok{"re"}\NormalTok{, }\StringTok{"bl"}\NormalTok{] + [}\StringTok{"gr"}\NormalTok{] }\CommentTok{# list concatenation}
\NormalTok{h = [}\StringTok{"re"}\NormalTok{] * }\DecValTok{5} \CommentTok{# repeat a list}
\NormalTok{[}\StringTok{"re"}\NormalTok{, }\StringTok{"bl"}\NormalTok{].index(}\StringTok{"re"}\NormalTok{) }\CommentTok{# returns index of "re"}
\StringTok{"re"} \NormalTok{in [}\StringTok{"re"}\NormalTok{, }\StringTok{"bl"}\NormalTok{] }\CommentTok{# true if "re" in list}
\DataTypeTok{sorted}\NormalTok{([}\DecValTok{3}\NormalTok{, }\DecValTok{2}\NormalTok{, }\DecValTok{1}\NormalTok{]) }\CommentTok{# returns sorted list}
\NormalTok{z = [}\StringTok{"red"}\NormalTok{] + [}\StringTok{"green"}\NormalTok{, }\StringTok{"blue"}\NormalTok{] }\CommentTok{# list concatenation}
\end{Highlighting}
\end{Shaded}

\subsubsection{Dictionaries}

\begin{Shaded}
\begin{Highlighting}[]
\NormalTok{a = \{}\StringTok{"red"}\NormalTok{: }\StringTok{"rouge"}\NormalTok{, }\StringTok{"blue"}\NormalTok{: }\StringTok{"bleu"}\NormalTok{, }\StringTok{"green"}\NormalTok{: }\StringTok{"vert"}\NormalTok{\} }\CommentTok{# dictionary}
\NormalTok{b = a[}\StringTok{"red"}\NormalTok{] }\CommentTok{# translate item}
\NormalTok{c = [value }\KeywordTok{for} \NormalTok{key, value in b.items()] }\CommentTok{# loop through contents}
\NormalTok{d = a.get(}\StringTok{"yellow"}\NormalTok{, }\StringTok{"no translation found"}\NormalTok{) }\CommentTok{# return default}
\end{Highlighting}
\end{Shaded}

\subsubsection{Strings}

\begin{Shaded}
\begin{Highlighting}[]
\NormalTok{a = }\StringTok{"red"} \CommentTok{# assignment}
\NormalTok{char = a[}\DecValTok{2}\NormalTok{] }\CommentTok{# access individual characters}
\StringTok{"red "} \NormalTok{+ }\StringTok{"blue"} \CommentTok{# string concatenation}
\StringTok{"1, 2, three"}\NormalTok{.split(}\StringTok{","}\NormalTok{) }\CommentTok{# split string into list}
\StringTok{"."}\NormalTok{.join([}\StringTok{"1"}\NormalTok{, }\StringTok{"2"}\NormalTok{, }\StringTok{"three"}\NormalTok{]) }\CommentTok{# concatenate list into string}
\end{Highlighting}
\end{Shaded}

\subsubsection{Operators}

\begin{Shaded}
\begin{Highlighting}[]
\NormalTok{a = }\DecValTok{2} \CommentTok{# assignment}
\NormalTok{a += }\DecValTok{1} \NormalTok{(*=, /=) }\CommentTok{# change and assign}
\DecValTok{3} \NormalTok{+ }\DecValTok{2} \CommentTok{# addition}
\DecValTok{3} \NormalTok{/ }\DecValTok{2} \CommentTok{# integer division (python2) or float division (python3)}
\DecValTok{3} \NormalTok{// }\DecValTok{2} \CommentTok{# integer division}
\DecValTok{3} \NormalTok{* }\DecValTok{2} \CommentTok{# multiplication}
\DecValTok{3} \NormalTok{** }\DecValTok{2} \CommentTok{# exponent}
\DecValTok{3} \NormalTok{% }\DecValTok{2} \CommentTok{# remainder}
\DataTypeTok{abs}\NormalTok{() }\CommentTok{# absolute value}
\DecValTok{1} \NormalTok{== }\DecValTok{1} \CommentTok{# equal}
\DecValTok{2} \NormalTok{> }\DecValTok{1} \CommentTok{# larger}
\DecValTok{2} \NormalTok{< }\DecValTok{1} \CommentTok{# smaller}
\DecValTok{1} \NormalTok{!= }\DecValTok{2} \CommentTok{# not equal}
\DecValTok{1} \NormalTok{!= }\DecValTok{2} \NormalTok{and }\DecValTok{2} \NormalTok{< }\DecValTok{3} \CommentTok{# logical AND}
\DecValTok{1} \NormalTok{!= }\DecValTok{2} \NormalTok{or }\DecValTok{2} \NormalTok{< }\DecValTok{3} \CommentTok{# logical OR}
\NormalTok{not }\DecValTok{1} \NormalTok{== }\DecValTok{2} \CommentTok{# logical NOT}
\NormalTok{a in b }\CommentTok{# test if a is in b}
\NormalTok{a is b }\CommentTok{# test if objects point to the same memory (id)}
\end{Highlighting}
\end{Shaded}

\subsubsection{Control Flow}

\paragraph{if/elif/else}

\begin{Shaded}
\begin{Highlighting}[]
\NormalTok{a, b = }\DecValTok{1}\NormalTok{, }\DecValTok{2}
\KeywordTok{if} \NormalTok{a + b == }\DecValTok{3}\NormalTok{:}
    \KeywordTok{print} \StringTok{'True'}
\KeywordTok{elif} \NormalTok{a + b == }\DecValTok{1}\NormalTok{:}
    \KeywordTok{print} \StringTok{'False'}
\KeywordTok{else}\NormalTok{:}
    \KeywordTok{print} \StringTok{'?'}
\end{Highlighting}
\end{Shaded}

\paragraph{for}

\begin{Shaded}
\begin{Highlighting}[]
\NormalTok{a = [}\StringTok{'red'}\NormalTok{, }\StringTok{'blue'}\NormalTok{,}
     \StringTok{'green'}\NormalTok{]}
\KeywordTok{for} \NormalTok{color in a:}
    \KeywordTok{print} \NormalTok{color}
\end{Highlighting}
\end{Shaded}

\paragraph{while}

\begin{Shaded}
\begin{Highlighting}[]
\NormalTok{number = }\DecValTok{1}
\KeywordTok{while} \NormalTok{number < }\DecValTok{10}\NormalTok{:}
    \KeywordTok{print} \NormalTok{number}
    \NormalTok{number += }\DecValTok{1}
\end{Highlighting}
\end{Shaded}

\paragraph{break}

\begin{Shaded}
\begin{Highlighting}[]
\NormalTok{number = }\DecValTok{1}
\KeywordTok{while} \OtherTok{True}\NormalTok{:}
    \KeywordTok{print} \NormalTok{number}
    \NormalTok{number += }\DecValTok{1}
    \KeywordTok{if} \NormalTok{number > }\DecValTok{10}\NormalTok{:}
        \KeywordTok{break}
\end{Highlighting}
\end{Shaded}

\paragraph{continue}

\begin{Shaded}
\begin{Highlighting}[]
\KeywordTok{for} \NormalTok{i in }\DataTypeTok{range}\NormalTok{(}\DecValTok{20}\NormalTok{):}
    \KeywordTok{if} \NormalTok{i % }\DecValTok{2} \NormalTok{== }\DecValTok{0}\NormalTok{:}
        \KeywordTok{continue}
    \KeywordTok{print} \NormalTok{i}
\end{Highlighting}
\end{Shaded}

\subsubsection{Functions, Classes, Generators, Decorators}

\paragraph{Function}

\begin{Shaded}
\begin{Highlighting}[]
\KeywordTok{def} \NormalTok{myfunc(a1, a2):}
    \KeywordTok{return} \NormalTok{x}

\NormalTok{x = my_function(a1,a2)}
\end{Highlighting}
\end{Shaded}

\paragraph{Class}

\begin{Shaded}
\begin{Highlighting}[]
\KeywordTok{class} \NormalTok{Point(}\DataTypeTok{object}\NormalTok{):}
    \KeywordTok{def} \OtherTok{__init__}\NormalTok{(}\OtherTok{self}\NormalTok{, x):}
        \OtherTok{self}\NormalTok{.x = x}
    \KeywordTok{def} \OtherTok{__call__}\NormalTok{(}\OtherTok{self}\NormalTok{):}
        \KeywordTok{print} \OtherTok{self}\NormalTok{.x}

\NormalTok{x = Point(}\DecValTok{3}\NormalTok{)}
\end{Highlighting}
\end{Shaded}

\paragraph{Generators}

\begin{Shaded}
\begin{Highlighting}[]
\KeywordTok{def} \NormalTok{firstn(n):}
    \NormalTok{num = }\DecValTok{0}
    \KeywordTok{while} \NormalTok{num < n:}
        \KeywordTok{yield} \NormalTok{num}
        \NormalTok{num += }\DecValTok{1}

\NormalTok{x = [}\KeywordTok{for} \NormalTok{i in firstn(}\DecValTok{10}\NormalTok{)]}
\end{Highlighting}
\end{Shaded}

\paragraph{Decorators}

\begin{Shaded}
\begin{Highlighting}[]
\KeywordTok{class} \NormalTok{myDecorator(}\DataTypeTok{object}\NormalTok{):}
    \KeywordTok{def} \OtherTok{__init__}\NormalTok{(}\OtherTok{self}\NormalTok{, f):}
        \OtherTok{self}\NormalTok{.f = f}
    \KeywordTok{def} \OtherTok{__call__}\NormalTok{(}\OtherTok{self}\NormalTok{):}
        \KeywordTok{print} \StringTok{"call"}
        \OtherTok{self}\NormalTok{.f()}

\OtherTok{@myDecorator}
\KeywordTok{def} \NormalTok{my_funct():}
    \KeywordTok{print} \StringTok{'func'}

\NormalTok{my_func()}
\end{Highlighting}
\end{Shaded}

\subsection{NumPy}

\subsubsection{array initialization}

\begin{Shaded}
\begin{Highlighting}[]
\NormalTok{np.array([}\DecValTok{2}\NormalTok{, }\DecValTok{3}\NormalTok{, }\DecValTok{4}\NormalTok{]) }\CommentTok{# direct initialization}
\NormalTok{np.empty(}\DecValTok{20}\NormalTok{, dtype=np.float32) }\CommentTok{# single precision array with 20 entries}
\NormalTok{np.zeros(}\DecValTok{200}\NormalTok{) }\CommentTok{# initialize 200 zeros}
\NormalTok{np.ones((}\DecValTok{3}\NormalTok{,}\DecValTok{3}\NormalTok{), dtype=np.int32) }\CommentTok{# 3 x 3 integer matrix with ones}
\NormalTok{np.eye(}\DecValTok{200}\NormalTok{) }\CommentTok{# ones on the diagonal}
\NormalTok{np.zeros_like(a) }\CommentTok{# returns array with zeros and the same shape as a}
\NormalTok{np.linspace(}\DecValTok{0}\NormalTok{., }\DecValTok{10}\NormalTok{., }\DecValTok{100}\NormalTok{) }\CommentTok{# 100 points from 0 to 10}
\NormalTok{np.arange(}\DecValTok{0}\NormalTok{, }\DecValTok{100}\NormalTok{, }\DecValTok{2}\NormalTok{) }\CommentTok{# points from 0 to <100 with step width 2}
\NormalTok{np.logspace(-}\DecValTok{5}\NormalTok{, }\DecValTok{2}\NormalTok{, }\DecValTok{100}\NormalTok{) }\CommentTok{# 100 logarithmically spaced points between 1e-5 and 1e2}
\NormalTok{np.copy(a) }\CommentTok{# copy array to new memory}
\end{Highlighting}
\end{Shaded}

\subsubsection{reading/ writing files}

\begin{Shaded}
\begin{Highlighting}[]
\NormalTok{np.fromfile(fname/}\DataTypeTok{object}\NormalTok{, dtype=np.float32, count=}\DecValTok{5}\NormalTok{) }\CommentTok{# read binary data from file}
\NormalTok{np.loadtxt(fname/}\DataTypeTok{object}\NormalTok{, skiprows=}\DecValTok{2}\NormalTok{, delimiter=}\StringTok{","}\NormalTok{) }\CommentTok{# read ascii data from file}
\end{Highlighting}
\end{Shaded}

\subsubsection{array properties and operations}

\begin{Shaded}
\begin{Highlighting}[]
\NormalTok{a.shape }\CommentTok{# a tuple with the lengths of each axis}
\DataTypeTok{len}\NormalTok{(a) }\CommentTok{# length of axis 0}
\NormalTok{a.ndim }\CommentTok{# number of dimensions (axes)}
\NormalTok{a.sort(axis=}\DecValTok{1}\NormalTok{) }\CommentTok{# sort array along axis}
\NormalTok{a.flatten() }\CommentTok{# collapse array to one dimension}
\NormalTok{a.conj() }\CommentTok{# return complex conjugate}
\NormalTok{a.astype(np.int16) }\CommentTok{# cast to integer}
\NormalTok{np.argmax(a, axis=}\DecValTok{2}\NormalTok{) }\CommentTok{# return index of maximum along a given axis}
\NormalTok{np.cumsum(a) }\CommentTok{# return cumulative sum}
\NormalTok{np.}\DataTypeTok{any}\NormalTok{(a) }\CommentTok{# True if any element is True}
\NormalTok{np.}\DataTypeTok{all}\NormalTok{(a) }\CommentTok{# True if all elements are True}
\NormalTok{np.argsort(a, axis=}\DecValTok{1}\NormalTok{) }\CommentTok{# return sorted index array along axis}
\end{Highlighting}
\end{Shaded}

\subsubsection{indexing}

\begin{Shaded}
\begin{Highlighting}[]
\NormalTok{a = np.arange(}\DecValTok{100}\NormalTok{) }\CommentTok{# initialization with 0 - 99}
\NormalTok{a[: }\DecValTok{3}\NormalTok{] = }\DecValTok{0} \CommentTok{# set the first three indices to zero}
\NormalTok{a[}\DecValTok{1}\NormalTok{: }\DecValTok{5}\NormalTok{] = }\DecValTok{1} \CommentTok{# set indices 1-4 to 1}
\NormalTok{a[start:stop:step] }\CommentTok{# general form of indexing/slicing}
\NormalTok{a[}\OtherTok{None}\NormalTok{, :] }\CommentTok{# transform to column vector}
\NormalTok{a[[}\DecValTok{1}\NormalTok{, }\DecValTok{1}\NormalTok{, }\DecValTok{3}\NormalTok{, }\DecValTok{8}\NormalTok{]] }\CommentTok{# return array with values of the indices}
\NormalTok{a = a.reshape(}\DecValTok{10}\NormalTok{, }\DecValTok{10}\NormalTok{) }\CommentTok{# transform to 10 x 10 matrix}
\NormalTok{a.T }\CommentTok{# return transposed view}
\NormalTok{np.transpose(a, (}\DecValTok{2}\NormalTok{, }\DecValTok{1}\NormalTok{, }\DecValTok{0}\NormalTok{)) }\CommentTok{# transpose array to new axis order}
\NormalTok{a[a < }\DecValTok{2}\NormalTok{] }\CommentTok{# returns array that fulfills elementwise condition}
\end{Highlighting}
\end{Shaded}

\subsubsection{boolean arrays}

\begin{Shaded}
\begin{Highlighting}[]
\NormalTok{a < }\DecValTok{2} \CommentTok{# returns array with boolean values}
\NormalTok{np.logical_and(a < }\DecValTok{2}\NormalTok{, b > }\DecValTok{10}\NormalTok{) }\CommentTok{# elementwise logical and}
\NormalTok{np.logical_or(a < }\DecValTok{2}\NormalTok{, b > }\DecValTok{10}\NormalTok{) }\CommentTok{# elementwise logical or}
\NormalTok{-a }\CommentTok{# invert boolean array}
\NormalTok{np.invert(a) }\CommentTok{# invert boolean array}
\end{Highlighting}
\end{Shaded}

\subsubsection{elementwise operations and math functions}

\begin{Shaded}
\begin{Highlighting}[]
\NormalTok{a * }\DecValTok{5} \CommentTok{# multiplication with scalar}
\NormalTok{a + }\DecValTok{5} \CommentTok{# addition with scalar}
\NormalTok{a + b }\CommentTok{# addition with array b}
\NormalTok{a / b }\CommentTok{# division with b (np.NaN for division by zero)}
\NormalTok{np.exp(a) }\CommentTok{# exponential (complex and real)}
\NormalTok{np.sin(a) }\CommentTok{# sine}
\NormalTok{np.cos(a) }\CommentTok{# cosine}
\NormalTok{np.arctan2(y,x) }\CommentTok{# arctan(y/x)}
\NormalTok{np.arcsin(x) }\CommentTok{# arcsin}
\NormalTok{np.radians(a) }\CommentTok{# degrees to radians}
\NormalTok{np.degrees(a) }\CommentTok{# radians to degrees}
\NormalTok{np.var(a) }\CommentTok{# variance of array}
\NormalTok{np.std(a, axis=}\DecValTok{1}\NormalTok{) }\CommentTok{# standard deviation}
\end{Highlighting}
\end{Shaded}

\subsubsection{inner / outer products}

\begin{Shaded}
\begin{Highlighting}[]
\NormalTok{np.dot(a, b) }\CommentTok{# inner matrix product: a_mi b_in}
\NormalTok{np.einsum(}\StringTok{"ijkl,klmn->ijmn"}\NormalTok{, a, b) }\CommentTok{# einstein summation convention}
\NormalTok{np.}\DataTypeTok{sum}\NormalTok{(a, axis=}\DecValTok{1}\NormalTok{) }\CommentTok{# sum over axis 1}
\NormalTok{np.}\DataTypeTok{abs}\NormalTok{(a) }\CommentTok{# return array with absolute values}
\NormalTok{a[}\OtherTok{None}\NormalTok{, :] + b[:, }\OtherTok{None}\NormalTok{] }\CommentTok{# outer sum}
\NormalTok{a[}\OtherTok{None}\NormalTok{, :] * b[}\OtherTok{None}\NormalTok{, :] }\CommentTok{# outer product}
\NormalTok{np.outer(a, b) }\CommentTok{# outer product}
\NormalTok{np.}\DataTypeTok{sum}\NormalTok{(a * a.T) }\CommentTok{# matrix norm}
\end{Highlighting}
\end{Shaded}

\subsubsection{interpolation, integration}

\begin{Shaded}
\begin{Highlighting}[]
\NormalTok{np.trapz(y, x=x, axis=}\DecValTok{1}\NormalTok{) }\CommentTok{# integrate along axis 1}
\NormalTok{np.interp(x, xp, yp) }\CommentTok{# interpolate function xp, yp at points x}
\end{Highlighting}
\end{Shaded}

\subsubsection{fft}

\begin{Shaded}
\begin{Highlighting}[]
\NormalTok{np.fft.fft(y) }\CommentTok{# complex fourier transform of y}
\NormalTok{np.fft.fftfreqs(}\DataTypeTok{len}\NormalTok{(y)) }\CommentTok{# fft frequencies for a given length}
\NormalTok{np.fft.fftshift(freqs) }\CommentTok{# shifts zero frequency to the middle}
\NormalTok{np.fft.rfft(y) }\CommentTok{# real fourier transform of y}
\NormalTok{np.fft.rfftfreqs(}\DataTypeTok{len}\NormalTok{(y)) }\CommentTok{# real fft frequencies for a given length}
\end{Highlighting}
\end{Shaded}

\subsubsection{rounding}

\begin{Shaded}
\begin{Highlighting}[]
\NormalTok{np.ceil(a) }\CommentTok{# rounds to nearest upper int}
\NormalTok{np.floor(a) }\CommentTok{# rounds to nearest lower int}
\NormalTok{np.}\DataTypeTok{round}\NormalTok{(a) }\CommentTok{# rounds to neares int}
\end{Highlighting}
\end{Shaded}

\subsubsection{random variables}

\begin{Shaded}
\begin{Highlighting}[]
\NormalTok{np.random.normal(loc=}\DecValTok{0}\NormalTok{, scale=}\DecValTok{2}\NormalTok{, size=}\DecValTok{100}\NormalTok{) }\CommentTok{# 100 normal distributed random numbers}
\NormalTok{np.random.seed(}\DecValTok{23032}\NormalTok{) }\CommentTok{# resets the seed value}
\NormalTok{np.random.rand(}\DecValTok{200}\NormalTok{) }\CommentTok{# 200 random numbers in [0, 1)}
\NormalTok{np.random.uniform(}\DecValTok{1}\NormalTok{, }\DecValTok{30}\NormalTok{, }\DecValTok{200}\NormalTok{) }\CommentTok{# 200 random numbers in [1, 30)}
\NormalTok{np.random.random_integers(}\DecValTok{1}\NormalTok{, }\DecValTok{15}\NormalTok{, }\DecValTok{300}\NormalTok{) }\CommentTok{# 300 random integers between [1, 10]}
\end{Highlighting}
\end{Shaded}

\subsection{Matplotlib}

\subsubsection{figures and axes}

\begin{Shaded}
\begin{Highlighting}[]
\NormalTok{fig = plt.figure(figsize=(}\DecValTok{5}\NormalTok{, }\DecValTok{2}\NormalTok{), facecolor=}\StringTok{"black"}\NormalTok{) }\CommentTok{# initialize figure}
\NormalTok{ax = fig.add_subplot(}\DecValTok{3}\NormalTok{, }\DecValTok{2}\NormalTok{, }\DecValTok{2}\NormalTok{) }\CommentTok{# add second subplot in a 3 x 2 grid}
\NormalTok{fig, axes = plt.subplots(}\DecValTok{5}\NormalTok{, }\DecValTok{2}\NormalTok{, figsize=(}\DecValTok{5}\NormalTok{, }\DecValTok{5}\NormalTok{)) }\CommentTok{# return fig and array of axes in a 5 x 2 grid}
\NormalTok{ax = fig.add_axes([left, bottom, width, height]) }\CommentTok{# manually add axes at a certain position}
\end{Highlighting}
\end{Shaded}

\subsubsection{figures and axes properties}

\begin{Shaded}
\begin{Highlighting}[]
\NormalTok{fig.suptitle(}\StringTok{"title"}\NormalTok{) }\CommentTok{# big figure title}
\NormalTok{fig.subplots_adjust(bottom=}\FloatTok{0.1}\NormalTok{, right=}\FloatTok{0.8}\NormalTok{, top=}\FloatTok{0.9}\NormalTok{, wspace=}\FloatTok{0.2}\NormalTok{,}
\NormalTok{hspace=}\FloatTok{0.5}\NormalTok{) }\CommentTok{# adjust subplot positions}
\NormalTok{fig.tight_layout(pad=}\FloatTok{0.1}\NormalTok{,h_pad=}\FloatTok{0.5}\NormalTok{, w_pad=}\FloatTok{0.5}\NormalTok{, rect=}\OtherTok{None}\NormalTok{) }\CommentTok{# adjust}
\NormalTok{subplots to fit perfectly into fig}
\NormalTok{ax.set_xlabel() }\CommentTok{# set xlabel}
\NormalTok{ax.set_ylabel() }\CommentTok{# set ylabel}
\NormalTok{ax.set_xlim(}\DecValTok{1}\NormalTok{, }\DecValTok{2}\NormalTok{) }\CommentTok{# sets x limits}
\NormalTok{ax.set_ylim(}\DecValTok{3}\NormalTok{, }\DecValTok{4}\NormalTok{) }\CommentTok{# sets y limits}
\NormalTok{ax.set_title(}\StringTok{"blabla"}\NormalTok{) }\CommentTok{# sets the axis title}
\NormalTok{ax.}\DataTypeTok{set}\NormalTok{(xlabel=}\StringTok{"bla"}\NormalTok{) }\CommentTok{# set multiple parameters at once}
\NormalTok{ax.legend(loc=}\StringTok{"upper center"}\NormalTok{) }\CommentTok{# activate legend}
\NormalTok{ax.grid(}\OtherTok{True}\NormalTok{, which=}\StringTok{"both"}\NormalTok{) }\CommentTok{# activate grid}
\NormalTok{bbox = ax.get_position() }\CommentTok{# returns the axes bounding box}
\NormalTok{bbox.x0 + bbox.width }\CommentTok{# bounding box parameters}
\end{Highlighting}
\end{Shaded}

\subsubsection{plotting routines}

\begin{Shaded}
\begin{Highlighting}[]
\NormalTok{ax.plot(x,y, }\StringTok{"-o"}\NormalTok{, c=}\StringTok{"red"}\NormalTok{, lw=}\DecValTok{2}\NormalTok{, label=}\StringTok{"bla"}\NormalTok{) }\CommentTok{# plots a line}
\NormalTok{ax.scatter(x,y, s=}\DecValTok{20}\NormalTok{, c=color) }\CommentTok{# scatter plot}
\NormalTok{ax.pcolormesh(xx,yy,zz, shading=}\StringTok{"gouraud"}\NormalTok{) }\CommentTok{# fast colormesh function}
\NormalTok{ax.colormesh(xx,yy,zz, norm=norm) }\CommentTok{# slower colormesh function}
\NormalTok{ax.contour(xx,yy,zz, cmap=}\StringTok{"jet"}\NormalTok{) }\CommentTok{# contour line plot}
\NormalTok{ax.contourf(xx,yy,zz, vmin=}\DecValTok{2}\NormalTok{, vmax=}\DecValTok{4}\NormalTok{) }\CommentTok{# filled contours plot}
\NormalTok{n, bins, patch = ax.hist(x, }\DecValTok{50}\NormalTok{) }\CommentTok{# histogram}
\NormalTok{ax.imshow(matrix, origin=}\StringTok{"lower"}\NormalTok{, extent=(x1, x2, y1, y2)) }\CommentTok{# show image}
\NormalTok{ax.specgram(y, FS=}\FloatTok{0.1}\NormalTok{, noverlap=}\DecValTok{128}\NormalTok{, scale=}\StringTok{"linear"}\NormalTok{) }\CommentTok{# plot a spectrogram}
\end{Highlighting}
\end{Shaded}

\end{document}
